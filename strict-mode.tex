\chapter{Striktný a superstriktný mód}

\label{kap:striktny_mod} % id kapitoly pre prikaz ref

V tejto kapitole popisujem pre striktný a superstriktný mód dôvody vzniku, implementáciu a zmeny, ktoré zavádzajú.

\section{Striktný mód}
Striktný mód je vlastnosť pridaná vo verzii ECMAScript 5, ktorá obmedzuje použitie určitých vlastností jazyka, ktoré sú považované za náchylné k chybám, a pridáva zmeny, ktoré robia kód v striktnom móde bezpečnejší.

Kontkrétne sa jedná o tieto zmeny~\cite{EcmaScript}:
\begin{itemize}
\item Medzi rezervované slová sa pridávajú $implements$, $interface$, $let$, $package$, $private$, $protected$, $public$, $static$ a $yield$, lebo sa očakáva, že budú použité v budúcich verziách a nemôžu byť preto použité ako názvy funkcií a premenných.
\item Priradenie do neexistujúcej premennej už danú globálnu premennú nevytvorí, ale spôsobí výnimku ReferenceError. Tým sa ľahšie odhalia chyby, keď programátor omylom spravil chybu v názve premennej a hodnotu priradí do odlišnej premennej.
\item Nie je možné použiť číselné literály, ktoré začínajú nulou. Všetky internetové prehliadače týmto spôsobom definujú čísla so základom 8, ale nie je to súčasť štandardu EcmaScript 5.
\item Nie je možné použiť výraz $with$. Tento výraz umožňuje pristupovať k vlastnostiam objektu, ktorý je argument výrazu, priamo bez nutnosti písať prefix objektu.
  % radšej dať ukážku
\end{itemize}

\subsection{Spôsob použitia}

\subsection{Zmeny}

\section{Superstriktný mód}
